\documentclass[{../RL_for_QSP.tex}]{subfiles}
%\usepackage{float}

\begin{document}
    \section{Introduction}
    \label{sec:INT}

A central requirement in order to engineer functional quantum computers is the ability to control quantum bits (qubits) robustly, flexibly, and cost-effectively with classical control systems. A fundamental problem in quantum control is quantum state preparation: a dynamical process that involves setting up one or more qubits in a desired configuration. One potential avenue towards improved quantum state preparation is the use of reinforcement learning. Reinforcement learning is a branch of machine learning in which a computer agent learns how best to perform a task using feedback from interactions with its environment. Reinforcement learning has been applied to a variety of quantum physics problems with great success \cite{2018zhang} \cite{fosel2018reinforcement} \cite{halverson2019branes} \cite{nautrup2019optimizing} \cite{melnikov2018active}. In 2018 and 2019, M. Bukov et al. and Zheng An and D. L. Zhou established the potential for quantum state preparation using reinforcement learning \cite{bukov2018reinforcement} \cite{an2019deep}. In a comparative study on quantum state preparation, X.-M. Zhang et al. demonstrated the advantages of reinforcement learning compared to other machine learning methods in terms of scalability and efficiency \cite{2019zhang}. Despite the favourable characteristics of the reinforcement learning approach, performance challenges remain when increasing the set of potential controls and the number of controls applied between the start and end state \cite{2019zhang}. Recently, a reinforcement learning framework was developed to incorporate correction for control errors; however, multiple sources of error such as approximation errors and environmental defects were not considered \cite{niu2019universal}. Moving forward, there are improvements to be made by tailoring reinforcement learning algorithms specifically for quantum state preparation. There is also more work to be done investigating the ability of reinforcement learning to find control solutions that are robust enough to withstand experimental imperfections. The objective of our research is to design and implement a reinforcement learning algorithm which will discover a sequence of discrete controls to bring one qubit from an initial state toward a desired state. The algorithm will maximize fidelity between the final state achieved by the control and the desired state. 

\end{document}