\documentclass[{../RL_for_QSP.tex}]{subfiles}

\begin{document}
    \section{Problem Description}
    \label{sec:PD}

% \subsection{Problem Statement}
% The goal is to design and implement a reinforcement learning algorithm which will discover a sequence of discrete controls to bring one (or more) qubits from an initial state toward a desired state. The algorithm will maximize fidelity between the final state achieved by the control and the desired state. 

\subsection{Model}
The previous section has provided the background necessary to present the stochastic control problem we aim to solve using reinforcement learning. Consider the problem of preparing a quantum state $\ket{\psi_f}$ from an initial state $\ket{\psi_i}$ in a finite time $\tau$ in $n$ steps. We study the Hamiltonian introduced in Eqn. \ref{spinhaml} for spin in a magnetic field. At each time $t$, a constant magnetic field $h_t = (h_t^x,h_t^z)$ is applied for the time step $\Delta t = \frac{\tau}{n}$ and some dephasing noise $\eta_t$ drawn from a zero-mean Gaussian distribution is present. Together this yields the Hamiltonion in Eqn. \ref{modelhaml}.
\begin{equation}\label{modelhaml}
     H_t = \eta_t\sigma_z -h_t^x \sigma_x - h_t^z\sigma_z
\end{equation}

Overall, the state evolves according to the model in Eqn. \ref{model}.

\begin{equation}\label{model}
     \ket{\psi_{t+1}} = e^{-i H_t \Delta t}\ket{\psi_t}
\end{equation}

The set of states is given by 
$$\mathbb{X} := \left\{ \ket{\psi} = \cos(\frac{\theta}{2})\ket{0} + e^{i\phi}\sin(\frac{\theta}{2})\ket{1} : 0 \leq \theta \leq \pi, 0 \leq \phi \leq 2 \pi \right\}$$ 
which correspond to points on the Bloch sphere. In practice, there is a maximum and minumum possible magnetic field strength within which the control can be tuned so the set of controls can be simplified to
$$\mathbb{U} := \{ h = (h^x,h^z) : h_i \in \{h_{min}, h_{min} + \Delta, h_{min} + 2\Delta, \hdots , h_{max}\}, \: i = x,z\}$$
where $\Delta = (h_{max} - h_{min})/M$, so that each component of the control field can take $M + 1$ values including $h_{min}$ and $h_{max}$ for some $M \in \mathbb{Z}_{>0}$.

The fidelity between the two states $\ket{\psi_0}$ and $\ket{\psi_1}$ is given by in Eqn. \ref{F}.
\begin{equation}\label{F}
     F(\psi_0, \psi_1) = |\braket{\psi_0}{\psi_1}|^2
\end{equation}

Consider the cost function in Eqn. \ref{cost}, which rewards fidelity between the current state $\ket{\psi_t}$ and the target state $\ket{\psi_f}$.
\begin{equation}\label{cost}
     c(\ket{\psi_t},h_t) = \begin{cases} 
     100F(\psi_t, \psi_f)^3 & 0 \leq F(\psi_t, \psi_f) < 0.99 \\
     5000 & 0.99 \leq F(\psi_t, \psi_f) \leq 1
     \end{cases}
\end{equation}

Altogether, the model in Eqn. \ref{model}, the state space $\mathbb{X}$, the action space $\mathbb{U}$, and the cost function in Eqn. \ref{cost} constitue a Markov control problem.

\subsection{Controllability}
% Introduction to Quantum Control and Dynamics pp. 84
% Topological properties of reachable sets and the control of quantum bits pp. 215

Before proceeding with the algorithm design, we first verify the controllability of the system. In many situations, the Schrödinger equation introduced in Section \ref{sec:BI} takes the form of the system in Eqn. \ref{shrctrl}, where $H_0, \hdots, H_m$ are $n \times n$ Hermitian matrices.
\begin{equation}\label{shrctrl}
     i \hslash \frac{\partial \ket{\psi}}{\partial t} =  \left( H_0  + \sum_{i=1}^m H_i u_i(t) \right) \ket{\psi}
\end{equation}

Such a system is (pure state) controllable if for every pair of initial and final states, $\ket{\psi_i}$ and $\ket{\psi_f}$, there exist control functions $u_1(t), \hdots, u_m(t)$ and a time $\tau > 0$ such that the solution of Eqn. \ref{shrctrl} at time $\tau$, with initial condition $\ket{\psi_i}$, is $\ket{\psi(\tau)}$ = $\ket{\psi_f}$ \cite{d2007introduction}. 

Eqn. \ref{shrctrl} can be transformed into the equivalent system given in Eqn. \ref{shrctrl2}.
\begin{equation}\label{shrctrl2}
     \frac{\partial \ket{\psi}}{\partial t} = A \ket{\psi} + \sum_{i=1}^m B_i \ket{\psi} u_i(t)
\end{equation}

Under the assumption that $u_1, \hdots, u_m$ are piecewsie continuous functions, there is an equivalent characterization of controllability. Namely, the system is controllable if the matrices $B_1, \hdots, B_m$ generate the Lie-algebra of $n \times n$ skew-Hermitian matrices with zero trace \cite{d2007introduction}. 

For the specific Hamiltonian introduced in Eqn. \ref{modelhaml}, 

$$A = -i \eta_t \sigma_z \text{ and } B_1 = i \sigma_x = \begin{pmatrix} 0 & i \\ i & 0\end{pmatrix} \text{ and } B_2 = i \sigma_z = \begin{pmatrix} i & 0 \\ 0 & -i\end{pmatrix}.$$

Since $B_1$ and $B_2$ are skew-Hermitian with zero trace and linearly independent, they form a basis for the Lie-algebra of $2 \times 2$ skew-Hermitian matrices with zero trace (for details see \cite{aless2010topo}). As a result, the overall system in Eqn. \ref{model} is controllable.

% \subsection{Formulation as a Markov Decision Process}
% The Markov decision process underlying this problem is as follow:
% % https://arxiv.org/pdf/1902.02157.pdf pp. 6
% % https://arxiv.org/pdf/1705.00565.pdf pp. 3

% \begin{itemize}
%   \item a set of states, $\mathbb{X} := \{ (\theta, \phi) : 0 \leq \theta \leq \pi, 0 \leq \phi \leq 2 \pi \}$ which correspond to points on the Bloch sphere,
%   \item a set of control actions, $\mathbb{U} := \{h : \mathbb{R} \to \{0,1\} \times \{0,1\}\}$,
%   \item a transition kernel, $\mathcal{T} \in \mathcal{P}(\mathbb{X}|\mathbb{X}\times\mathbb{U})$, and
%   \item an initial distribution, $\nu$. 
% \end{itemize}

% % a set of control actions, $\mathbb{U} := \{h_{min}, h_{min} + \Delta, h_{min} + 2\Delta, \hdots , h_{max}\}$ where $\Delta = (h_{max} - h_{min})/M$, so that the control field can take $M + 1$ values including $h_{min}$ and $h_{max}$ for some $M \in \mathbb{Z}_{>0}$

% The fidelity between the current state $\ket{\psi}$ and the target state $\ket{\psi_f}$ is given by:
% \begin{align*}
%      F(\psi, \psi_f) &= |\braket{\psi}{\psi_f}|^2
% \end{align*}
  
% The reward function is based on fidelity.

% \begin{equation}
%      r(x_t,u_t,x_{t+1}) = \begin{cases} 
%      100F^3 & 0 \leq F < 0.99 \\
%      5000 & 0.99 \leq F \leq 1
%      \end{cases}
% \end{equation}

% An algorithm will be designed solve the Markov decision problem:
% $$\max_{\gamma \in \Gamma}E^{\gamma} \left[ \sum_{t = 0}^T r(x_t, u_t, x_{t + 1}) \right].$$

\end{document}